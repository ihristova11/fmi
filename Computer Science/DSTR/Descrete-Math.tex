\documentclass{article}
\usepackage[utf8]{inputenc}
\usepackage[bulgarian]{babel}

\title{Дискретна математика - подготовка за изпит}
\author{Ирина Христова}
\date{Октомври 2019}

\begin{document}

\maketitle

\section{Тема 1}
\textbf{Съждителна логика - прости съждения, логически съюзи, съставни съждения. Основни свойства на логическите съюзи. Таблици на истинност. Еквивалентност на съставни съждения. Методи за доказателство на еквивалентност: табличен метод и метод с еквивалентни преобразувания. Основи на предикатната логика - дефиниция на предикат, универсален и екзистенциален квантор. Свойства на отрицанието в предикатната логика.}
\newline\newline
1. Съждителна логика
\newline
\textit{Логика - наука за правенето на валидни изводи и правилни разсъждения}
\newline\newline
\textit{Съждение - просто разказвателно изречение, което може да е истина или лъжа}
\newline\newline



\end{document}
