\documentclass{article}
\usepackage[utf8]{inputenc}
\usepackage[bulgarian]{babel}

\title{Релации}
\author{Ирина Христова}
\date{Ноември 2019}

\begin{document}

\maketitle

\section{Дефиниция}


\section{Видове релации}
\subsection{рефлексивна}
\forall a \in A, (a,a) \in R 

\subsection{симетрична}
\forall a, b \in A, a\ne b, (a,b) \in R \rightarrow (b,a) \in R

\subsection{транзитивна}
\forall a, b, c \in A, (a,b) \in R, (b,c) \in R \rightarrow (a,c) \in R

\subsection{антирефлексивна}
\forall a \in A, (a,a) \notin R

\subsection{антисиметрична}
\forall a,b \in A, a \ne b, (a,b) \in R \rightarrow (b,a) \notin R (или еквивалентно \forall a,b \in A, (a,b) \in R, (b,a) \in R \rightarrow a = b

\subsection{силно антисиметрична}
\forall a,b \in A, a \ne b, точно едно от (a,b) \in R или (b,a) \notin R 

\section{Дефиниция}
дефиниции за рефлексивно, симетрично и транзитивно затваряне ... 



\end{document}
